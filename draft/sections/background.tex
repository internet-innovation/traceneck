\section{Background}

\subsection{Active Queue Management (AQM)}

AQM techniques have been an active area of research and deployment for the past few decades to reduce latency and bufferbloat~\cite{bufferbloat} and to ensure fairness and coexistence 
among TCP flows on the shared network links. Traditional AQM algorithms have been built to run in conjunction with TCP congestion control algorithms, which rely only on packet loss as a signal for congestion.

To handle the bursty nature of TCP, these AQM techniques are equipped with large data buffers to prevent excessive packet drops due to these bursts. 
However, bufferbloat arises when queue length grows unbounded, specifically if the buffers are increasingly large, the packets end up being queued in much deeper queues, leading to excessive queuing latency.
latency can often build up as a result of individual bufferbloats at multiple routers on the network path. 

AQM technique such as Random Early Detection~\cite{red}, CoDel~\cite{codel} and FQ-CoDel~\cite{fqcodel} have been
primarily designed to reduce latency and bufferbloat, by actively managing the queue lengths and dropping packets before the queue becomes full.
Other techniques such as Stochastic Fair Queuing (SFQ)~\cite{sfq} and Deficit Round Robin (DRR)~\cite{drr} aim to provide
fairness among competing flows by ensuring equal bandwidth allocation.
More recently, AQM techniques such as Proportional Integral controller Enhanced (PIE)~\cite{pie} and Low Latency, Low Loss, Scalable Throughput (L4S)~\cite{l4s} 
have been proposed to provide low latency and high throughput for modern applications such as video streaming and online gaming.


\siddhant{Should mention something about L4S here? It is positioned as a new AQM which tries to universally solve bufferbloat and fairness but we don't evaluate it.}

\subsection{Speedtest Measurement Tools}

\siddhant{Add details for each of these tools.}

\begin{itemize}
    \item M-Lab NDT: The Measurement Lab's Network Diagnostic Tool (NDT) is a widely used tool for measuring network performance, including bandwidth, latency, and packet loss.
    \item Ookla Speedtest: Ookla's Speedtest is a popular web-based tool that provides users with real-time information about their internet connection speed, including download and upload speeds.
    \item Apple Speedtest: Apple's Speedtest app is designed for iOS devices and provides users with a simple way to test their internet connection speed.
\end{itemize}

