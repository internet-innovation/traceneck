% This is samplepaper.tex, a sample chapter demonstrating the
% LLNCS macro package for Springer Computer Science proceedings;
% Version 2.21 of 2022/01/12
%

\newcommand{\paul}[1]{\ignorespaces}
\newcommand{\fb}[1]{\ignorespaces}
\newcommand{\nick}[1]{\ignorespaces}
\newcommand{\jonatas}[1]{\textcolor{blue}{[JM: #1]}}
\newcommand{\siddhant}[1]{\textcolor{blue}{[SR: #1]}}
\newcommand{\taveesh}[1]{\textcolor{orange}{[TS: #1]}}

\documentclass[runningheads]{llncs}
%
\usepackage[T1]{fontenc}
% T1 fonts will be used to generate the final print and online PDFs,
% so please use T1 fonts in your manuscript whenever possible.
% Other font encondings may result in incorrect characters.
%
\usepackage{graphicx}
\usepackage{xcolor}
% Used for displaying a sample figure. If possible, figure files should
% be included in EPS format.
%
% If you use the hyperref package, please uncomment the following two lines
% to display URLs in blue roman font according to Springer's eBook style:
%\usepackage{color}
%\renewcommand\UrlFont{\color{blue}\rmfamily}
%\urlstyle{rm}
%
\begin{document}
%
\title{Characterizing the Impact of Active Queue Management on Speedtest Measurements}
%
%\titlerunning{Abbreviated paper title}
% If the paper title is too long for the running head, you can set
% an abbreviated paper title here
%

\author{Anonymous}
% \author{First Author\inst{1}\orcidID{0000-1111-2222-3333} \and
% Second Author\inst{2,3}\orcidID{1111-2222-3333-4444} \and
% Third Author\inst{3}\orcidID{2222--3333-4444-5555}}
%
\authorrunning{F. Author et al.}
% First names are abbreviated in the running head.
% If there are more than two authors, 'et al.' is used.
%
% \institute{Princeton University, Princeton NJ 08544, USA \and
% Springer Heidelberg, Tiergartenstr. 17, 69121 Heidelberg, Germany
% \email{lncs@springer.com}\\
% \url{http://www.springer.com/gp/computer-science/lncs} \and
% ABC Institute, Rupert-Karls-University Heidelberg, Heidelberg, Germany\\
% \email{\{abc,lncs\}@uni-heidelberg.de}}
%
\maketitle              % typeset the header of the contribution
%

\begin{abstract}

Present day speed test tools measure peak throughput, but often fail to capture the user-perceived responsiveness of a network connection under load. Recently, platforms such as Ookla's Speedtest.net and Cloudflare have introduced metrics such as ``latency under load'' or ``working latency'' to fill this gap. Yet, the sensitivity of these metrics to basic network configurations such as Active Queue Management (AQM) remains poorly understood. In this work, we conduct an empirical study of the impact of AQM on speed test measurements in a laboratory setting. Using controlled experiments, we compare the distribution of throughput, latency, and latency under load measurements across different AQM schemes, including CoDel, FQ-CoDel and Stochastic Fair Queuing (SFQ). On comparing the results with a standard drop-tail baseline, we find that \taveesh{add the main punchline here.} These results highlight the critical role of AQM in shaping how emerging latency metrics should be interpreted, and underscore the need for careful calibration of speed test platforms before their results are used to guide policy or regulatory outcomes.

\keywords{Speed Test \and Active Queue Management \and Responsiveness \and Bufferbloat}

\end{abstract}

\section{Introduction}

Internet performance has historically been summarized using a single number: ``speed'' \cite{bauer2010understanding,midoglu2018monroenettestconfigurabletooldissecting,feamster2019internetspeedmeasurementcurrent}. Despite the widespread utility, the user-perceived Quality of Experience (QoE) for many applications (e.g., video-conferencing, gaming, cloud collaboration) is governed less by peak bandwidth and more by latency under load. To address this, measurement providers have recently begun introducing ``latency under load'' (LUL) or ``responsiveness'' metrics, which attempt to capture how queuing delays increase during download and upload activity.

However, the interpretation and use of these metrics has not been standardized. For instance, Ookla defines ``working latency'' as the increase in round-trip time (RTT) under load compared to the unloaded RTT, measured during a speed test \cite{CeroWRT_speedtests}. Apple uses a different metric, called round trips per minute (RPM) under load, which counts the number of round trips completed during a fixed time interval while the connection is saturated \cite{ietf-ippm-responsiveness-07}. Further, these tests have been known to discard outliers that often correspond to glitches that users typically notice during real-time applications such as video conferencing and streaming \cite{CeroWRT_speedtests}. As a result, users and regulators are left with incomplete pictures of what causes an Internet connection to be unresponsive, and how it can be mitigated. A central, unanswered question is how traditional metrics such as throughput and latency, and new metrics such as LUL and RPM, behave in the presence or absence of active queue management (AQM) algorithms such as FQ-CoDel, which were explicitly designed to maintain low latency under load \cite{hoilandbufferbloat}.

In this paper, we investigate how the empirical distribution of modern speed test measurement results shifts when an AQM is deployed. Rather than reporting only typical throughput (e.g., mean) and latency values (e.g., $90^{th}$ percentile, median), we analyze full distributions: the tails, the spikes, and metrics similar to ``glitches per minute'' \cite{CeroWRT_speedtests} that are most relevant to real-time applications. Our goal is to empirically characterize the difference between unmanaged queues and AQM-enabled network paths, and to highlight how this difference is (or is not) reflected in widely deployed measurement platforms. By doing so, we aim to inform both test designers and network operators of the gaps between the status quo of Internet measurement and the actual experience of end-users.

\taveesh{Methods summary and contributions go here.}

\siddhant{We also need some stronger justifications for why we are doing this study. 
Maybe something like: "While AQM has been widely studied in the context of TCP performance, its impact on speedtest measurements remains underexplored.
Secondly, AQM deployment is steadily increasing (some numbers from Jason/Comcast article/any other source)
Understanding this relationship is crucial for both network operators and end-users to accurately assess and improve their internet experience."}

\begin{thebibliography}{8}

\end{thebibliography}

\end{document}
